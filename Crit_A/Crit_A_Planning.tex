\documentclass[a4paper, 12pt]{article}

\usepackage[margin=1in]{geometry}

\setlength{\parindent}{0em}
\setlength{\parskip}{1em}

\title{Crit A - Planning}
\date{}

\begin{document}
\maketitle

\section{Defining the Problem}

The \emph{client}, Dr. Venia Papaspyrou, is the IBDP coordinator at my school. At the
beginning of each year she is responsible for creating the timetable which will be used for
the rest of the academic year. 

\emph{Currently}, at the beginning of each academic year, she asks each prospective IB1
student for their preferred subjects and then manually creates the timetable. This takes a
lot of time, given the number of students and subjects. It is also error-prone, as it is
possible that, for example, some students might end up with overlapping subjects, meaning
that further adjustments will have to be made. 

The schedule that will be made has the following characteristics:
%
\begin{itemize}
    \item 4 periods per day, with five schooldays per week
    \item SL subjects have 2 periods per week and HL subjects have 3
    \item For some subjects, such as mathematics, standard and higher level are taught as
        completely different subjects --- students of math SL and math HL never share a
        classroom for math --- while for others, such as physics, SL and HL have two shared
        periods, and there is a third period in which only HL participates. 
\end{itemize}
%

\section{Rationale and proposed solution}

After discussing with the client about her requirements, I decided that the best solution
would be an application, where she would input the required data (student subject
preferences, maximum class sizes, other limitations concerning the students or teachers,
etc.), and the application would create an appropriate timetable given those limitations. 

This would greatly reduce the time required for creating the timetable, since it would take
at most a few minutes, rather than the days that are required for creating the timetable
manually. It would also lead to far fewer errors since the possibility of human error would
be eliminated. 

Dr. Papaspyrou already uses Microsoft Excel for storing the data regarding the timetable,
meaning that she is already familiar with the program. Therefore, it would help with the
learnability of the application if she could continue to use the same format for inputting
the data as well as for receiving the output. For this reason, I decided to have the
application receive its data from a Microsoft Excel spreadsheet, meaning that the user will
input which students chose which subjects in a spreadsheet with a specified form, open the
application, choose the file they have created and then the application will output a second
Microsoft Excel spreadsheet containing the final timetable.

The application will be created in python, mainly because of my familiarity with the
language as well as its simplicity. This will allow for the application to be developed much
quicker, since I will not have to take time to learn a new language. I will also mean that I
will be able to use advanced features of the language, since I would not have experience
with equivalent features in other languages. Aside from my own familiarity, Python is also
one of the most popular programming languages in the world, and there is therefore a
plethora of ready-made libraries for various different uses which will also help speed up
development. 

\section{Success criteria}

\begin{enumerate}
    \item The interface is simple and easy to use
    \item The input of the program is in Microsoft Excel form
    \item The layout of the spreadsheet used for input is intuitive and a skeleton is
        provided with the program
    \item The input data is validated and, if there is an error, the user is told which cell
        it is in
    \item The output is produced within a reasonable amount of time
    \item The output is in Microsoft Excel form
    \item The timetables produced allow students to attend all subjects they have chosen
        (i.e. no student has two subjects on the same period)
    \item The program can handle each case for subjects with a standard and higher level,
        meaning:
        \begin{enumerate}
            \item completely separate SL and HL classes
            \item two shared periods of SL and HL and one extra period for HL
        \end{enumerate}
\end{enumerate}

\end{document}
