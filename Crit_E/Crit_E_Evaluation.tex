\documentclass[a4paper, 12pt]{article}

\usepackage[margin=1in]{geometry}

\title{Criterion E - Evaluation}
\author{}
\date{}

\begin{document}
\maketitle

\section{Meeting the Success Criteria}
\begin{enumerate}
    \item The interface is simple and easy to use. - \textbf{Done, only consists of two
        buttons, and the user did not need any guidance to understand what each does.}
    \item The input of the program is in Microsoft Excel form. - \textbf{Done.}
    \item A skeleton of the input spreadsheet is provided with the program. - \textbf{Done.}
    \item The input data is validated and, if there is an error, the user is told which cell
        it is in. - \textbf{Both done. When the user used the program she was quickly able to
        find a missing cell in the data file when she saw the error message.}
    \item The output is produced within a reasonable amount of time. - \textbf{Done, output
        is usually produced within a few seconds, though for certain inputs, it may take
        longer, or even continue running without halting.}
    \item The output is in Microsoft Excel form. - \textbf{Done.}
    \item The timetables produced allow students to attend all subjects they have chosen
        (i.e. no student has two subjects on the same period). - \textbf{Done.}
\end{enumerate}

\section{Recommendations for future improvement}

While the success criteria were achieved, there are still improvements that could be made to
the application. The most significant problem found in the application occurred in cases
where the data provided was too complex for the solver to create a timetable. The program
would keep on running until the user manually stopped it. This is a problem as the user
does not know whether the program has frozen or whether it will actually produce a result.
The simplest solution to this problem would be to add a \emph{timeout} value to the program,
so that, when the processing takes more than a certain amount of time the user is notified,
and it is suggested that they quit the program and try a simpler set of data. A better
solution would be to add a progress bar showing how long the processing will take. This is
not possible with the constraint solver library used currently, but could be possible with
an alternative library, or if I were to implement the constraint solving algorithm myself. 

The client has also mentioned that having to move out of Excel to create the timetable is
somewhat inconviniet, therefore, in order to further improve ease of use, it could also be
possible to replace the interface of the program with a Microsoft Excel plugin, so that the
user could start the solver directly from the interface of Excel, without having to open a
second application. This could be achieved by creating a Microsoft Excel plugin that starts
the application in the background with the currently open file as input and stores its
output to a separate worksheet in the file.

\end{document}
