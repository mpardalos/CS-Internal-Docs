\documentclass[12pt]{article}

\usepackage[a4paper, margin=1in]{geometry}
\usepackage{graphicx}
    \graphicspath{{../media/}} 
\usepackage[
    defernumbers=true,
    citestyle=authortitle,
    autocite=footnote
]{biblatex}
    \addbibresource{../bibliography.bib}
\usepackage[
    colorlinks,
    linkcolor=black,
    citecolor=black,
    urlcolor=blue
]{hyperref}
\usepackage{float}
\usepackage{mdframed}
\usepackage[textfont=it]{caption}
\usepackage{listings}
\usepackage{minted}
    \setminted{
        bgcolor=LightGray,
        linenos,
        frame=lines,
        framesep=5mm,
        fontsize=\footnotesize
    }
\usepackage[svgnames]{xcolor}

\newcommand{\code}[1]{\texttt{\color{Grey}#1}}

\setlength{\parskip}{1em}
\setlength{\parindent}{0em}

\title{Appendix C - Generators}
\date{}


\begin{document}

\maketitle
\vspace{-5em}

Generators are a feature of the python language that is used extensively in my program and
so, knowledge of them is necessary to understand the function of the program. 

"Generators functions allow you to declare a function that behaves like an iterator, i.e. it
can be used in a for loop."\autocite{generators}. When a function has to return a list of
values, it is usually much simpler to create a generator function than to directly return a
list. Consider the following example: \vspace{-6mm}
%
\begin{figure}[H]
    \caption{}\vspace{-6mm}
    \begin{minted}{python}
def count(limit):
    i = 0
    while i <= limit:
        yield i
        i += 1

for num in count(5):
    print(num)
    \end{minted}
\end{figure}
%
This will output:\vspace{-5mm}
%
\begin{verbatim}
1
2
3
4
5
\end{verbatim}
%
When a function has the \code{yield} keyword inside it, it automatically becomes a
generator and can be used in a \code{for} loop like in the example. What happens in the
\code{for} loop in the example is the following: When the loop begins, the function is run
until a \code{yield} statement is found. Then, the loop variable (\code{i}) is set to
the value passed to \code{yield} statement (which is 0 for the first iteration), and the
body of the loop is executed. Control is passed back to the generator, and line 5 is
executed, the generator keeps control until another yield statement is hit (which will be on
line 4, since we are in a \code{while} loop in the generator), and the cycle continues.
When either a \code{return} statement or (in this case) the end of the generator is hit,
the loop stops.

\printbibliography
\end{document}
